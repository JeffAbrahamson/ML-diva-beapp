\input ../talk-header.tex
\title{Machine Learning et IA pour marketing et commerciaux}
\subtitle{Comment vendre l'IA (et pas seulement les LLM)}

\begin{document}

\begin{frame}
  \titlepage
\end{frame}


\begin{frame}{Objectifs}
  \begin{itemize}
  \item Comprendre les bénéfices clés de l'IA et savoir les
    communiquer de manière claire et accessible.
  \item Illustrer les concepts avec des exemples concrets et des cas
    d'utilisation pertinents pour l'industrie des participants.
  \item Sensibiliser les clients sur les résultats attendus et les
    limitations possibles des solutions IA.
  \end{itemize}
\end{frame}

\begin{frame}
\frametitle{Brève introduction sur l'importance de bien vendre l'IA}

\textbf{Contexte :}
\begin{itemize}
\item L'IA est une technologie révolutionnaire qui peut transformer de
  nombreux aspects des affaires et de la société.
\item Cependant, elle est souvent mal comprise ou entourée de hype, ce
  qui peut créer des attentes irréalistes ou des réticences.
  \item Une solution à 100~\% n'est ni réaliste ni requise.
\end{itemize}
\end{frame}

\begin{frame}{Pourquoi bien vendre l'IA est crucial}

\textbf{1. Établir la confiance :}
\begin{itemize}
\item Les clients doivent comprendre comment l'IA peut apporter de la
  valeur ajoutée à leur entreprise.
    \item Une communication claire et transparente sur les capacités
      et les limitations de l'IA aide à bâtir une relation de
      confiance.
    \item C'est important que le client comprenne la méthodologie
      itérative.
    \item Selon le client, on peut parler de la solution aléatoire
      (comme référence).
\end{itemize}
\end{frame}

\begin{frame}{Pourquoi bien vendre l'IA est crucial :}

\textbf{2. Différenciation sur le marché :}
\begin{itemize}
    \item L'IA peut être un facteur différenciant majeur par rapport à la concurrence.
    \item Savoir bien vendre l'IA permet de positionner l'entreprise
      comme un leader innovant.
\end{itemize}
\end{frame}

\begin{frame}{Pourquoi bien vendre l'IA est crucial :}

\textbf{3. Alignement des attentes :}
\begin{itemize}
    \item Clarifier ce que l'IA peut et ne peut pas faire permet de gérer les attentes des clients.
    \item Cela réduit le risque de déceptions et augmente la satisfaction client.
\end{itemize}
\end{frame}

\begin{frame}{Pourquoi bien vendre l'IA est crucial :}

\textbf{4. Valorisation des investissements :}
\begin{itemize}
\item Les projets d'IA peuvent représenter un investissement
  important.
\item Expliquer les bénéfices tangibles (ROI, amélioration des
  processus, gain de temps) permet de justifier ces
  investissements.
\item Parler de l'amélioration continue (modèle, données, algorithmes,
  apprentissage humain, \dots)
\end{itemize}
\end{frame}

\begin{frame}{Pourquoi bien vendre l'IA est crucial :}

\textbf{5. Adoption et intégration :}
\begin{itemize}
\item Une bonne communication sur les avantages et l'impact de l'IA
  favorise l'acceptation et l'adoption par les utilisateurs finaux.
\item Cela facilite également l'intégration de l'IA dans les processus
  existants.
\item Répétons : \underbar{processus}
\end{itemize}
\end{frame}

\begin{frame}{MVP}
  \solo{Itération}
\end{frame}

\begin{frame}{Exemples concrets :}

\textbf{Presse :}
\begin{itemize}
\item Un projet de résumé automatisé d'articles en montrant comment il
  peut augmenter l'engagement des lecteurs en leur fournissant des
  informations clés rapidement.
\end{itemize}

\textbf{E-commerce :}
\begin{itemize}
\item Recommandations personnalisées qui augmentent les ventes de
  manière significative chez d'autres clients.
\end{itemize}
\end{frame}

\begin{frame}{Conclusion :}
\begin{itemize}
\item Vendre l'IA efficacement :
  \begin{itemize}
  \item une compréhension claire des besoins du client
  \item des bénéfices concrets pour leur entreprise
  \item une communication transparente sur les capacités et les
    limites de la technologie
  \end{itemize}
\item Une bonne stratégie de vente de l'IA combine des explications
  accessibles, des exemples pertinents et une gestion réaliste des
  attentes.
\end{itemize}
\end{frame}

\begin{frame}{Des bénéfices de l'IA}
\begin{itemize}
    \item Amélioration de l'efficacité et de la productivité.
    \item Personnalisation accrue des services et produits.
    \item Prédiction et analyse avancée.
\end{itemize}
\end{frame}

\begin{frame}{Exemples concrets :}
\begin{itemize}
\item Presse : Résumés automatisés permettant une lecture rapide des
  nouvelles importantes.
\item E-commerce : Recommandations personnalisées augmentant les
  ventes.
\end{itemize}

\purple{(brainstorming)}
\end{frame}

\begin{frame}{Exemples et cas d'utilisation}
\begin{itemize}
    \item Présentation de cas d'utilisation pertinents :
    \begin{itemize}
    \item IoT : Optimisation des systèmes connectés pour réduire les
      coûts énergétiques.
    \item Santé : Diagnostic assisté par IA pour améliorer la
      précision et réduire le temps de diagnostic.
    \end{itemize}
  \item Discussion sur les résultats obtenus et les améliorations
    constatées.
\end{itemize}
\end{frame}

\begin{frame}{Discussion sur les limitations et les attentes réalistes}
\begin{itemize}
    \item Transparence sur les défis et les limitations de l'IA.
    \item Importance de la qualité des données et des modèles d'entraînement.
\end{itemize}
\end{frame}

\begin{frame}{Exemple concret}
\begin{itemize}
    \item Les algorithmes de matchmaking sur les sites de rencontre (et leurs limites).
\end{itemize}
\end{frame}

\talksection{Exercice}

\begin{frame}
\textbf{Objectif de l'exercice :}
\begin{itemize}
\item Élaborer une stratégie de vente d'un projet IA fictif adapté à
  un des domaines traités par l'entreprise.
\item Choisir un domaine (presse, IoT, e-commerce, santé, sites de rencontre, comptabilité).
\item Élaborer une courte présentation (5 minutes) pour vendre un
  projet IA en mettant en avant les bénéfices, des exemples concrets
  et en abordant les attentes réalistes.
\end{itemize}
\end{frame}

\begin{frame}{Points à aborder dans la présentation}
\begin{itemize}
    \item Problème spécifique à résoudre avec l'IA.
    \item Bénéfices attendus pour le client.
    \item Exemple concret d'utilisation.
    \item Limitations possibles et comment les gérer.
\end{itemize}
\end{frame}

\begin{frame}{Restitution}
\begin{itemize}
    \item Présenter sa stratégie de vente.
    \item Discussion et (auto-)feedback collectif
\end{itemize}
\end{frame}

\end{document}
