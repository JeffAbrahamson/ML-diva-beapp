\input ../talk-header.tex \title{Machine Learning et IA pour marketing
et commerciaux} \subtitle{Éthique et responsabilité}

\begin{document}

\begin{frame} \titlepage
\end{frame}


\begin{frame}{Introduction à l'éthique et la responsabilité en IA}
  \begin{itemize}
  \item Définition de la responsabilité dans le contexte du ML.
  \item Importance de l'éthique, de l'accessibilité, de l'inclusivité,
    et de l'éco-conception dans les projets ML.
  \item Introduction aux normes RGPD et leur importance pour la
    protection des données.
  \end{itemize}
\end{frame}

\begin{frame}{RGPD (en révision\dots)}
    \begin{itemize}
        \item {Consentement}
        \item {Intérêt légitime} (contrat, obligations légales, \dots)
    \end{itemize}
\end{frame}

\begin{frame}{RGPD}
  \begin{itemize}
  \item {Informations Claires} : informer sur la manière dont leurs données sont collectées, utilisées, stockées et partagées.
        \item {Accès et Rectification} : droit d'accès, de correction ou de suppression.
    \end{itemize}
\end{frame}

\begin{frame}{RGPD}
  Minimisation des Données
  \begin{itemize}
  \item {Données pertinentes} et nécessaires
  \item {Anonymisation et pseudonymisation}
  \end{itemize}
\end{frame}

\begin{frame}{RGPD}
  Droits des Utilisateurs

  \begin{itemize}
  \item {Droit à l'oubli}
  \item {Portabilité des données}
  \item {Opposition}
  \end{itemize}
\end{frame}

\begin{frame}{RGPD}
  Sécurité des Données
  \begin{itemize}
  \item {Protection des données}
  \item {Data breach}
  \end{itemize}
\end{frame}

\begin{frame}{RGPD}
  Responsabilité et documentation

  \begin{itemize}
  \item {DPO}
  \item {Documentation}
    \end{itemize}
\end{frame}

\begin{frame}{RGPD et l'IA/ML}
  \begin{itemize}
  \item {Biais et discrimination}
  \item {Explicabilité} \purple{\textit{(droit à une explication ?)}}
  \item {Impact sur la vie privée}
  \end{itemize}
\end{frame}

\begin{frame}{RGPD}
    \begin{itemize}
        \item {Transfert en dehors de l'UE}
    \end{itemize}
\end{frame}

\begin{frame}{RGPD}
\begin{itemize}
    \item {DPIA pour les Systèmes ML/IA}
    \item Identification et atténuation des risques pour la vie privée.
\end{itemize}
\end{frame}

\begin{frame}{RGPD}
  \begin{itemize}
  \item {Techniques de protection des données}
  \item Protection contre la ré-identification des individus.
  \end{itemize}
\end{frame}

\begin{frame}{RGPD}
  \begin{itemize}
  \item {Précautions pour les Données Sensibles}
    \item Conformité avec les exigences strictes du RGPD.
\end{itemize}
\end{frame}

\begin{frame}{RGPD}
  Décisions automatisées et profilage

  \begin{itemize}
  \item {Limites des décisions automatisées} \purple{\textit{(droit de
        ne pas être soumis à des décisions basées uniquement sur un
        traitement automatisé ?)}}
  \item Intervention humaine nécessaire pour certaines décisions importantes.
  \end{itemize}
\end{frame}

\begin{frame}{RGPD}
  Minimisation des données pour l'entraînement des modèles

  \begin{itemize}
  \item {Minimisation et pertinence}
  \end{itemize}
\end{frame}

\begin{frame}{RGPD}
  Sécurité des Données

  \begin{itemize}
  \item {Sécurité des modèles et des données}
  \item Protection contre les accès non autorisés et les cyberattaques.
\end{itemize}
\end{frame}

\begin{frame}{Éco-conception et optimisation des ressources}
  \begin{itemize}
  \item Explication de l'impact environnemental des technologies ML
    (énergie, matériel, etc.).
  \item Techniques pour réduire la consommation d'énergie (par
    exemple, optimisation des modèles, utilisation d'infrastructures
    plus vertes).
  \item Importance de l'éco-conception dans le cycle de vie du
    développement ML.
    \begin{itemize}
    \item Choix du modèle
    \item Quantité de données
    \end{itemize}
  \end{itemize}
\end{frame}

\begin{frame}{Accessibilité et inclusivité}
  \begin{itemize}
  \item Le ML peut aider à améliorer l'accessibilité (reconnaissance
    vocale, traduction automatique, etc.).
  \item Enjeux de l'inclusivité dans les jeux de données et les biais
    potentiels (Uber, sèche-mains, reconnaissance faciale, \dots).
  \end{itemize}
\end{frame}

\begin{frame}{Sécurité et protection des données}
  \begin{itemize}
  \item Principes de base de la sécurisation des données dans les
    projets ML.
  \item Impact du RGPD sur la collecte et l'utilisation des données.
  \item Stratégies pour sécuriser les données, y compris le
    chiffrement (au repos et en transit) et l'anonymisation.
  \end{itemize}
\end{frame}

\begin{frame}{Définition et contexte}
  \begin{itemize}
  \item {Introduction à la deanonymisation} (surtout avant entraînement)
  \item {Exemple} : taxis (mais pas que)
  \end{itemize}
\end{frame}

\begin{frame}{Techniques utilisées pour la deanonymisation}
  \begin{itemize}
  \item {Inférence de données} : des
    informations apparemment anonymes peuvent être croisées avec
    d'autres sources de données pour réidentifier des individus.
  \item {Limites de l'anonymisation} : Certaines méthodes
    d'anonymisation échouent --- par exemple, insuffisance de
    l'agrégation, réversibilité des hachages avec puissance de calcul
    élevée, \dots.
  \end{itemize}
\end{frame}

\begin{frame}{Stratégies de protection contre la deanonymisation}
  \begin{itemize}
  \item \textbf{Renforcement des techniques d'anonymisation} :
    Approfondir des méthodes avancées comme le "differential privacy",
    où l'ajout de bruit dans les données empêche la réidentification.
  \item \textbf{Contrôles d'accès et surveillance} : Mettre en place
    des politiques strictes de contrôle d'accès aux données et des
    audits réguliers pour détecter et prévenir les tentatives de
    deanonymisation.
  \item \textbf{Formation et sensibilisation} : Souligner l'importance
    de former le personnel impliqué dans les projets du ML à comprendre
    et à mettre en œuvre des pratiques de gestion des données
    sécurisées.
  \end{itemize}
\end{frame}

\talksection{Exercice}

\begin{frame}{Exercice}
  Choisir un projet et analyser comment le rendre plus sécure, plus
  respectueux de la vie privée et moins energivore.
\end{frame}

\begin{frame}{Étapes de l'exercice}
  \begin{itemize}
  \item Choisir un projet
  \item Identifier les enjeux spécifiques en matière d'éthique,
    d'accessibilité, d'éco-conception, et de protection des données.
  \item {Élaborer une solution}
    \begin{itemize}
    \item Proposer un modèle de machine learning adapté.
    \item Planifier des méthodes pour minimiser l'impact écologique
      (serveurs verts, optimisation du modèle).
    \item Développer des stratégies pour garantir l'accessibilité et
      l'inclusivité.
    \end{itemize}
  \end{itemize}
\end{frame}

\end{document}

