\input ../talk-header.tex \title{Machine Learning et IA pour marketing
et commerciaux} \subtitle{Choix de solution / technologie}

\begin{document}

\begin{frame}
  \titlepage
\end{frame}

\begin{frame}{Objectifs}
  Comprendre quand utiliser différentes approches (boîte noire, LLM,
  ML classique, ou solution sans ML) en fonction des besoins du projet
  et des contraintes spécifiques.
\end{frame}

\begin{frame}{Approche : Boîte noire}
\begin{itemize}
    \item Définition
    \item Avantages : performance souvent supérieure, moins de besoin en expertise technique.
    \item Inconvénients : manque de transparence, difficulté à expliquer les résultats.
\end{itemize}
\end{frame}

\begin{frame}{Approche : LLM (Large Language Models)}
\begin{itemize}
    \item Définition
    \item Avantages : capable de comprendre et générer du langage naturel, adaptabilité.
    \item Inconvénients : coût élevé en ressources, complexité, besoin de données massives.
\end{itemize}
\end{frame}

\begin{frame}{Approche : ML classique}
\begin{itemize}
    \item Définition
    \item Avantages : souvent plus compréhensible, (beaucoup) plus léger en ressources.
    \item Inconvénients : peut nécessiter plus de travail de préparation des données
\end{itemize}
\end{frame}

\begin{frame}{Approche : Solution sans ML}
\begin{itemize}
    \item Définition
    \item Avantages : simplicité, coût souvent moindre, transparence.
    \item Inconvénients : limité en termes de capacités d'apprentissage et d'adaptabilité.
\end{itemize}
\end{frame}

\begin{frame}{Critères de choix}
\begin{itemize}
    \item Nature du problème à résoudre.
    \item Disponibilité et qualité des données.
    \item Contraintes de temps et de budget.
    \item Besoins en transparence et explicabilité.
    \item Capacité de l'équipe à maintenir et à comprendre la solution.
\end{itemize}
\end{frame}

\talksection{Pas d'exercice}

\begin{frame}{Pas d'exercice}
  Le sujet demande la présence de profils techniques.
\end{frame}

\end{document}
