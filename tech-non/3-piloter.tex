\input ../talk-header.tex
\title{Machine Learning et IA pour marketing et commerciaux}
\subtitle{Comment piloter un projet avec le client}

\begin{document}

\begin{frame}
  \titlepage
\end{frame}

\begin{frame}{Objectifs}
  \begin{itemize}
  \item Développer une compréhension claire de l'intégration du ML et
    de l'IA dans les processus de développement de projets existants
  \item Focus sur la gestion des attentes et la collaboration efficace
    avec le client.
  \end{itemize}
\end{frame}

\begin{frame}{Présentation}
  \begin{itemize}
  \item \textbf{Définition et rappel} : Qu'est-ce que le ML et l'IA?
    Pourquoi sont-ils pertinents pour le développement d'applications
    modernes?
  \item \textbf{Importance de l'alignement avec le projet global} :
    Comment le ML/IA s'insère dans la vision globale du projet.
  \end{itemize}
\end{frame}

\begin{frame}{2. Étapes clés de l'intégration du ML/IA}

  \subsubsection{Cadrage du projet}
  \begin{itemize}
  \item Définir les objectifs clairs avec le client.
  \item Identifier les données disponibles et nécessaires.
  \item Évaluation de la faisabilité technique et des bénéfices attendus.
  \item Définir les mesures de réussites (même partiels).
  \end{itemize}
\end{frame}

\begin{frame}{2. Étapes clés de l'intégration du ML/IA (suite)}

  \subsubsection{Conception}
  \begin{itemize}
  \item Prototypage rapide et itérations.
  \item Choix des technologies et des outils.
  \item Définition des critères de succès (y compris partiel).
  \item Aider le client à percevoir les petites avancées progressives.
  \end{itemize}
\end{frame}

\begin{frame}{2. Étapes clés de l'intégration du ML/IA (suite)}

  \subsubsection{Développement et mise en œuvre}
  \begin{itemize}
  \item Intégration du modèle ML/IA dans l'application existante.
  \item Gestion des modifications et des mises à jour.
  \item Tests et validations avec des données réelles.
  \item Mise en place de systèmes d'évaluation contenue (« model drift »).
  \end{itemize}
\end{frame}

\begin{frame}{Model drift}
  \only<1>{
  \begin{itemize}
  \item \textbf{Dérive des Données :} Changements dans la distribution des
    données d'entrée sans changements dans la relation entre l'entrée
    et la cible.  \purple{\textit{(data drift}}
  \item \textbf{Dérive du Concept :} Changements dans la relation entre les
    données d'entrée et la cible, affectant la manière dont le modèle
    doit interpréter les données.  \purple{\textit{(concept drift)}}
  \item \textbf{Dérive du Modèle :} Le terme général décrivant la
    diminution des performances du modèle, qui peut être causée par la
    dérive des données, la dérive du concept ou d'autres facteurs.
    \purple{\textit{(model drift)}}
  \end{itemize}
}
\only<2>{
  \purple{\textbf{Exemple de dérive du modèle}}

  \purple{Imaginons un modèle de recommandation de produits en ligne
    qui, au fil du temps, voit ses performances
    diminuer. Initialement, il recommandait avec précision des
    produits que les utilisateurs achetaient souvent. Cependant, avec
    le temps, les préférences des utilisateurs changent, de nouveaux
    produits apparaissent, et la compétition introduit de nouvelles
    stratégies marketing. Le modèle, entraîné sur des données plus
    anciennes, devient moins pertinent et ses recommandations sont de
    moins en moins suivies, ce qui montre une dérive du modèle.}
}
\end{frame}

\begin{frame}{Dérive des données}
  \only<1>{
  La dérive des données, également connue sous le nom de changement de
  covariables, se produit lorsque les propriétés statistiques des
  données d'entrée (variables indépendantes) changent au fil du
  temps. Ce changement peut rendre le modèle moins efficace car il a
  été entraîné sur des données avec des caractéristiques
  différentes. Il existe deux principaux types de dérive des données :

  \begin{itemize}
  \item \textbf{Changement de la probabilité a priori :} La
    distribution de la variable cible reste la même, mais la
    distribution des caractéristiques d'entrée change.
    \purple{\textit{(prior probability shift)}}
  \item \textbf{Changement de covariables :} La relation entre les
    caractéristiques d'entrée et la variable cible reste la même, mais
    la distribution des caractéristiques d'entrée change.
    \purple{\textit{(covariate shift)}}
  \end{itemize}
}
\only<2>{
\purple{\textbf{Exemple de dérive des données}}

\purple{Supposons qu'une entreprise utilise un modèle pour prédire les
  ventes en fonction des données météorologiques. Initialement, le
  modèle a été entraîné avec des données de températures et de
  précipitations. Cependant, avec le changement climatique, les
  conditions météorologiques changent : les hivers deviennent plus
  doux et les étés plus chauds. La distribution des températures
  change par rapport aux données d'entraînement initiales, ce qui
  entraîne une dérive des données. Le modèle devient alors moins
  précis car il n'a pas été entraîné pour ces nouvelles conditions
  météorologiques.}
}
\end{frame}

\begin{frame}{Dérive du concept}
  \only<1>{
  La dérive du concept fait référence au changement dans la relation
  entre les données d'entrée et la variable cible au fil du temps. Ce
  changement peut se produire en raison de l'évolution des tendances,
  de nouveaux comportements ou de changements dans l'environnement qui
  affectent le processus sous-jacent modélisé. Il existe différents
  types de dérive du concept :

  \begin{itemize}
  \item \textbf{Dérive soudain :} Changements brusques dans la
    distribution des données sous-jacentes.
  \item \textbf{Dérive graduel :} Changements lents et continus au fil
    du temps.
  \item \textbf{Dérive répétitive :} Changements qui se produisent
    périodiquement.
  \item \textbf{Dérive incrémentale :} Petits changements progressifs
    qui s'accumulent au fil du temps.
  \end{itemize}
}
\only<2>{
  \purple{\textbf{Exemple de Dérive du Concept}}

  \purple{Imaginons un modèle de détection de fraudes pour une banque
    qui a été entraîné pour détecter les transactions frauduleuses
    basées sur des comportements spécifiques des
    utilisateurs. Cependant, les fraudeurs changent leurs méthodes
    pour éviter d'être détectés, adoptant de nouvelles techniques et
    comportements. La relation entre les caractéristiques d'entrée
    (par exemple, les montants des transactions, les heures de
    transaction) et la variable cible (fraude ou non-fraude)
    change. Le modèle initial, entraîné sur les anciens comportements
    des fraudeurs, devient moins efficace. C'est un exemple de dérive
    du concept.}  }
\end{frame}

\begin{frame}{3. Gestion des attentes et des risques}

  \begin{itemize}
  \item \textbf{Absence de garantie de résultat} :
    \begin{itemize}
    \item Comment communiquer efficacement les incertitudes et les risques.
    \item Importance de la transparence avec le client.
    \item Souligner l'intérêt des résultats partiels.
    \end{itemize}
  \item \textbf{Impacts sur les phases du projet} :
    \begin{itemize}
    \item Modifications du calendrier de développement.
    \item Besoins en formation et en support technique pour l'équipe projet.
    \item Recherche de littérature : existence de projets similaires pour comprendre ce qui a marché / pas marché dans le passé.
    \end{itemize}
  \end{itemize}
\end{frame}

\talksection{Exercice}

\begin{frame}
\textbf{Objectif} : Simuler la planification et la gestion d'un projet
d'intégration d'un système de recommandations basé sur l'IA dans une
application e-commerce existante.
\end{frame}

\begin{frame}{Étapes de l'exercice}

  \begin{enumerate}
  \item \textbf{Définition de vos rôles}
  \item \textbf{Cadrage} :
    \begin{itemize}
    \item Définition des objectifs du système de recommandations.
    \item Identification des sources de données disponibles.
    \item Discussion sur les attentes du client et établissement d'un
      plan de communication.
    \end{itemize}
  \item \textbf{Conception} :
    \begin{itemize}
    \item Créer un prototype simplifié de la solution (sur papier).
    \item Définition des indicateurs de performance pour évaluer la
      solution.
    \end{itemize}
  \item \textbf{Présentation au client} (simulé) :
  \item \textbf{Feedback et révision} :
    \begin{itemize}
    \item Retour sur les présentations / suggestions d'améliorations /
      discussion sur les défis rencontrés.
    \end{itemize}
  \end{enumerate}
\end{frame}

\begin{frame}{Conclusion}
  Résumons les leçons apprises, des défis spécifiques liés à
  l'intégration du ML/IA dans les projets et partageons des stratégies
  pour améliorer la collaboration avec le client dans des projets
  futurs.
\end{frame}

\end{document}
