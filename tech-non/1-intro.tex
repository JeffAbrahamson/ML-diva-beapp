\input ../talk-header.tex
\title{Machine Learning et IA pour marketing et commerciaux}
\subtitle{Introduction à l'IA et au ML}

\begin{document}

\begin{frame}
  \titlepage
\end{frame}

\talksection{Bienvenue}

\begin{frame}{Structure}
  \only<1>{
    Deux journées
    \begin{itemize}
    \item  29, 30 août
    \end{itemize}
  }
  \only<2>{
    \begin{itemize}
    \item Email \url{jeff@p27.eu}
    \item Github : \url{https://github.com/JeffAbrahamson/ML-diva-beapp}
    \end{itemize}
  }
  \only<3>{
    \begin{itemize}
    \item Introduction à l'IA et au ML
    \item Comment vendre l'IA
    \item Comment piloter un projet avec le client
    \item Introduction à l'éthique et la responsabilité en IA
    \item Estimation et gestion des coûts
    \item Choix de la solution appropriée
    \end{itemize}
  }
  \only<4-7>{
    Fréquents exercices en groupe.

    Dans un sens, toujours le même exercice.
    \only<5>{\textit{\purple{Ce sera un peu comme l'esprit de la
          psychothérapie (pour ceux qui la connaissent) : apprendre en
          faisant dans un contexte encadré.}}}
    \only<6-7>{\textit{\purple{Une partie intégrale de votre apprentissage.}}}

    Objectif : vous aider à réfléchir
    différemment. \only<7>{\textit{\purple{Avec nos parcours
          différents, nous apportons aussi des perspectives variées,
          ce qui enrichit notre discussion.}}}

    Je suis là pour remplir les trous côte ML.
  }
\end{frame}

\begin{frame}{Brainstorming}
  \solopb{\Large\bf\blue{Discuter comment on espère se servir\\ du ML et de l'IA}}
  \vfill
\end{frame}

\talksection{Introduction à l'IA et au ML}

\begin{frame}{Objectifs}
  \begin{itemize}
  \item Comprendre ce qu'est l'Intelligence Artificielle (IA) et le Machine Learning (ML).
  \item Appréhender les concepts de base et les différents types de solutions IA.
  \end{itemize}
\end{frame}

\begin{frame}{Définition et Historique}
  \begin{itemize}
  \item Qu'est-ce que l'IA ? Brève histoire de l'IA et évolution du domaine.
  \item Qu'est-ce que le ML ? Différence entre IA et
    ML. \only<2>{\\ \purple{Pour nous, ML=algos et ANN, IA=LLM ou marketing}}
  \item IA faible v \gray{fort}
  \end{itemize}
\end{frame}

\begin{frame}{IA Faible (``Narrow AI'')}
  \only<1>{
  Définition :

  \bigskip
  \begin{quote}
    L'IA faible, aussi appelée IA étroite ou spécialisée,
    est conçue pour accomplir des tâches spécifiques. Elle excelle dans
    une seule tâche ou un ensemble limité de tâches.
  \end{quote}
}
\only<2>{
  Characteristiques
  \begin{itemize}
  \item \textbf{Spécialisée :} Conçue pour résoudre des problèmes
    particuliers (par exemple, reconnaissance faciale, traitement du
    langage naturel, jeu d'échecs).
  \item \textbf{Limitée :} Ne possède pas de conscience, de
    compréhension ou de capacité à raisonner au-delà de ses paramètres
    programmés.
  \end{itemize}
  }
  \only<3>{
    Exemples

    \begin{itemize}
    \item Siri d'Apple ou Alexa d'Amazon pour l'assistance vocale.
    \item Algorithmes de recommandation de Netflix ou Amazon.
    \item Voitures autonomes.
    \item Systèmes de diagnostic médical utilisant l'apprentissage machine.
    \end{itemize}
  }
  \only<4>{
    Fonctionnement

    \begin{itemize}
    \item Utilise des algorithmes et des données pour accomplir des
      tâches spécifiques de manière efficace.
    \item Est très performante dans le domaine pour lequel elle est
      conçue, mais ne peut pas généraliser ses compétences à d'autres
      domaines.
    \end{itemize}
  }
\end{frame}

\begin{frame}{IA Forte (General AI)}
  \only<1>{
    \gray{Definition}

  \bigskip
  \begin{quote}
    \gray{L'IA forte, aussi appelée IA générale ou AGI (Artificial General
    Intelligence), fait référence à des systèmes d'IA qui possèdent
    des capacités cognitives humaines. Elle peut comprendre,
    apprendre, et appliquer ses connaissances de manière générale,
    similaire à un humain.}
  \end{quote}
}
\only<2>{
  \gray{Caractéristiques}

  \begin{itemize}
  \item \gray{\textbf{Polyvalente :} Capable de comprendre, apprendre
      et appliquer ses compétences à une variété de tâches de manière
      flexible.}
  \item \gray{\textbf{Consciente et Réfléchie :} Possède un niveau de
      conscience, de compréhension et de capacité à raisonner
      comparable à celui des humains.}
  \end{itemize}
}
\only<3>{
  \gray{Exemples}

  \begin{itemize}
  \item \gray{Hypothétiquement, une IA capable de comprendre et de
      faire n'importe quelle tâche humaine, comme écrire de la poésie,
      résoudre des problèmes complexes en physique, ou gérer des
      interactions sociales.}
  \item \gray{Aucun système actuel ne correspond à la définition de
      l'IA forte, elle reste un objectif théorique pour les
      chercheurs.}
  \end{itemize}
}
\only<4>{
  \gray{Fonctionnement}

  \begin{itemize}
  \item \gray{Nécessite des avancées significatives en compréhension
      des processus cognitifs humains.}
  \item \gray{Serait capable d'apprendre de manière autonome et de
      transférer ses connaissances d'un domaine à un autre sans
      intervention humaine directe.}
  \end{itemize}
}
\end{frame}

\begin{frame}{Comparaison}
  \only<1>{
    Capacité et Portée

  \begin{itemize}
  \item \textbf{IA Faible :} Très bonne dans des tâches spécifiques,
    mais incapable de sortir de son domaine d'expertise.
  \item \gray{\textbf{IA Forte :} En théorie, capable de performer
      n'importe quelle tâche cognitive humaine de manière flexible.}
  \end{itemize}
  }
  \only<2>{
    Niveau de Conscience

  \begin{itemize}
  \item \textbf{IA Faible :} Aucune conscience ou compréhension réelle de ses actions.
  \item \gray{\textbf{IA Forte :} Posséderait une conscience et une
      compréhension semblable à celle des humains (hypothétique).}
  \end{itemize}
  }
  \only<3>{
    Exemples Actuels

  \begin{itemize}
  \item \textbf{IA Faible :} Technologies couramment utilisées
    aujourd'hui (assistants vocaux, systèmes de recommandation,
    chatbots).
  \item \gray{\textbf{IA Forte :} Encore inexistante, reste un objectif
      de recherche à long terme.}
  \end{itemize}
} \only<4>{ L'IA faible est ce que nous utilisons et développons
  actuellement dans divers domaines, tandis que l'IA forte est une
  ambition futuriste qui, si réalisée, pourrait révolutionner notre
  compréhension et interaction avec les machines.  }
\end{frame}

\begin{frame}{Type de ML}
  \begin{itemize}
  \item Apprentissage supervisé, non supervisé et par renforcement
  \item Regression vs classification
  \end{itemize}
\end{frame}

\begin{frame}{Exemple : Presse}
  \textbf{Problématique :}
  Créer des résumés d'articles de presse pour les utilisateurs.

  \textbf{Approche :}
  \textbf{Classification :} Identifier les phrases clés à inclure dans le résumé.

  \textbf{Exemple :}
  \begin{itemize}
  \item \textbf{Étape 1 :} Utiliser un modèle de classification pour
    analyser les phrases d'un article et déterminer lesquelles sont
    les plus importantes.
  \item \textbf{Étape 2 :} Classifier chaque phrase comme "importante"
    ou "non importante".
  \item \textbf{Étape 3 :} Générer un résumé en sélectionnant les
    phrases classées comme importantes.
  \end{itemize}
  \end{frame}

  \begin{frame}{Exemple : IoT}
    \textbf{Problématique :}
    Optimiser la consommation d'énergie des appareils connectés dans une maison intelligente.

    \textbf{Approche :}
    \textbf{Régression} - Prédire la consommation future d'énergie.

    \textbf{Exemple :}
    \begin{itemize}
    \item \textbf{Étape 1 :} Collecter des données historiques sur la
      consommation d'énergie de différents appareils.
    \item \textbf{Étape 2 :} Utiliser un modèle de régression pour
      prédire la consommation future d'énergie en fonction de facteurs
      comme l'heure de la journée, la température, et les habitudes
      d'utilisation.
    \item \textbf{Étape 3 :} Ajuster automatiquement les paramètres des
      appareils pour minimiser la consommation d'énergie.
    \end{itemize}
\end{frame}

\begin{frame}{Exemple : E-commerce}
    \textbf{Problématique :}
    Recommander des produits aux utilisateurs en fonction de leurs achats passés.

    \textbf{Approche :}
    \textbf{Classification} - Identifier les catégories de produits préférées des utilisateurs.

    \textbf{Exemple :}
  \begin{itemize}
  \item \textbf{Étape 1 :} Analyser les données d'achat des
    utilisateurs.
  \item \textbf{Étape 2 :} Utiliser un modèle de classification pour
    déterminer les catégories de produits les plus susceptibles
    d'intéresser chaque utilisateur.
  \item \textbf{Étape 3 :} Utiliser ces informations pour recommander
    des produits similaires ou complémentaires.
  \end{itemize}
\end{frame}

\begin{frame}{Exemple : Santé}
    \textbf{Problématique :}
    Aider les médecins à diagnostiquer des maladies à partir d'images médicales.

    \textbf{Approche :}
    \textbf{Classification} - Identifier la présence de maladies dans les images.

    \textbf{Exemple :}
    \begin{itemize}
    \item \textbf{Étape 1 :} Utiliser des images médicales annotées
      pour entraîner un modèle de classification.
  \item \textbf{Étape 2 :} Le modèle apprend à distinguer entre les
    images présentant des signes de maladie et celles qui n'en
    présentent pas.
  \item \textbf{Étape 3 :} Le modèle est utilisé pour analyser de
    nouvelles images et fournir un diagnostic assisté.
  \end{itemize}
\end{frame}

\begin{frame}{Exemple : Sites de rencontre}
    \textbf{Problématique :}
    Trouver des correspondances entre utilisateurs en fonction de leurs préférences et caractéristiques.

    \textbf{Approche :}
    \textbf{Classification} - Prédire la compatibilité entre deux utilisateurs.

    \textbf{Exemple :}
  \begin{itemize}
  \item \textbf{Étape 1 :} Collecter des données sur les préférences et
    caractéristiques des utilisateurs.
  \item \textbf{Étape 2 :} Utiliser un modèle de classification pour
    prédire la compatibilité entre deux utilisateurs en fonction de
    leurs données.
  \item \textbf{Étape 3 :} Utiliser ces prédictions pour suggérer des
    correspondances aux utilisateurs.
  \end{itemize}
\end{frame}

\begin{frame}{Exemple : Comptabilité}
  \textbf{Problématique :} Identifier les transactions frauduleuses
  parmi des milliers de transactions financières.

  \textbf{Approche :} \textbf{Classification} - Classifier les
  transactions comme frauduleuses ou non frauduleuses.

  \textbf{Exemple :}
  \begin{itemize}
  \item \textbf{Étape 1 :} Utiliser des données historiques sur les
    transactions frauduleuses et non frauduleuses pour entraîner un
    modèle de classification.
  \item \textbf{Étape 2 :} Le modèle apprend à identifier les
    caractéristiques des transactions frauduleuses.
  \item \textbf{Étape 3 :} Analyser les nouvelles transactions avec le
    modèle pour détecter les fraudes potentielles en temps réel.
  \end{itemize}
\end{frame}

\begin{frame}{Processus de création d'un modèle ML}
  \begin{itemize}
   \item  Collecte des données
   \item  Préparation des données
   \item  Entraînement du modèle
   \item  Évaluation et ajustement
  \end{itemize}
\end{frame}

\begin{frame}{Limites du ML}
  \begin{itemize}
  \item Besoin de grandes quantités de données
  \item Problèmes de biais et d'équité
  \item Complexité et coût
  \end{itemize}
\end{frame}

\begin{frame}{Démystification}
  \begin{itemize}
  \item Ce n'est pas magique.
  \item C'est basé sur des algorithmes mathématiques.
  \end{itemize}
\end{frame}

\begin{frame}{Bien définir son problème}
  C'est ça, la statistique !
  \only<1>{
      \begin{itemize}
      \item Identifier (précisément~!) une question ou problème
      \item Collecter les données nécessaires
      \item Analyser les données
      \item En tirer des conclusions
      \end{itemize}
    }
  \only<2>{
      \begin{itemize}
      \item Définir (précisément~!) la question ou problème
      \item Collecter les données nécessaires
      \item Nettoyer les données
      \item Explorer les données
      \item Créer un (des) modèle(s)
      \item Communiquer les résultats
      \item Rendre le tout reproductible
      \end{itemize}
    }
\end{frame}

\begin{frame}{Biais}
  \cimghh{../tech-oui/images/bias-variance.png}
\end{frame}

\begin{frame}{Underfitting, overfitting}
  \cimg{../tech-oui/images/under-overfitting.png}
\end{frame}

\talksection{Exercice}

\begin{frame}{Brainstorming}
  Brainstorming
  \begin{itemize}
  \item Réfléchir à une application mobile développée par Beapp
  \item Identifier une fonctionnalité ou un problème spécifique où le
    ML pourrait apporter une valeur ajoutée
  \end{itemize}

  Analyse de Faisabilité
  \begin{itemize}
  \item Quelles sont les données nécessaires ?
  \item Discuter des avantages potentiels et des limitations possibles
    de l'implémentation de cette solution.
  \end{itemize}
\end{frame}

\end{document}
