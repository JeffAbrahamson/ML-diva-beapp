\input ../talk-header.tex \title{Machine Learning et IA pour marketing
et commerciaux} \subtitle{Estimation et gestion des coûts}

\begin{document}

\begin{frame}
  \titlepage
\end{frame}

\begin{frame}{Objectifs}
  \begin{itemize}
  \item Comprendre les différents types de coûts associés aux projets de ML/IA.
  \item Apprendre à estimer ces coûts de manière réaliste.
  \item Discuter des stratégies pour gérer les coûts et optimiser les ressources.
  \end{itemize}
\end{frame}

\begin{frame}{Introduction aux coûts de l'IA/ML}
  \begin{itemize}
  \item Différents types de coûts : développement, données,
    infrastructure, maintenance, dérivée du modèle.
  \item Importance de la prévision des coûts pour la viabilité du projet.
  \end{itemize}
\end{frame}

\begin{frame}{Méthodologies d'estimation des coûts}
  \begin{itemize}
  \item Méthode basée sur les phases du projet : exploration,
    développement, déploiement.
  \item Impact des choix technologiques sur les coûts : choix des
    plateformes, des outils.
  \item Coûts des données : acquisition, nettoyage, annotation.
  \end{itemize}
\end{frame}

\begin{frame}{Gestion et optimisation des coûts}
  \begin{itemize}
  \item Budgétisation efficace : allocation des ressources, prévisions
    ajustables.
  \item Utilisation de services cloud et solutions open-source pour
    réduire les coûts.
  \end{itemize}

  \only<2>{
    \bigskip
    \purple{Brainstorming de bonnes pratiques en termes de gestion de projet
      ML/IA.}}
\end{frame}

\begin{frame}{Dérivée du modèle}
  Il est essentiel d'aborder les coûts associés au « model drift »
  (dérive du modèle) et à la re-apprentissage des modèles, car cela a un
  impact significatif sur le cycle de vie et les coûts globaux d'un
  projet d'IA/ML.
\end{frame}

\begin{frame}{Introduction aux coûts de l'IA/ML}
  \begin{itemize}
  \item Présentation des différents types de coûts : développement,
    données, infrastructure, maintenance.
  \item L'importance de la prévision des coûts pour assurer la
    viabilité à long terme du projet.
  \end{itemize}
\end{frame}

\begin{frame}{Méthodologies d'estimation des coûts}
  \begin{itemize}
  \item Estimation des coûts par phases du projet : exploration,
    développement, déploiement.
  \item Impact des choix technologiques : plateformes, outils.
  \item Coûts liés aux données : acquisition, nettoyage, annotation.
  \end{itemize}
\end{frame}

\begin{frame}{Coûts de la dérivée du modèle et de la re-apprentissage des modèles}
  \begin{itemize}
  \item {Impact financier :} Coûts de surveillance continue, de
    diagnostic de la performance du modèle, et de la collecte de
    nouvelles données pertinentes.
  \item {Coûts de re-apprentissage :} Évaluation des coûts associés à la
    re-apprentissage périodique des modèles pour maintenir ou améliorer la
    performance, incluant les ressources computationnelles et
    humaines.
  \item {Stratégies de mitigation :} exemples de techniques pour
    minimiser le model drift : l'apprentissage en continu et
    l'utilisation de pipelines de données automatisés
  \end{itemize}
\end{frame}

\begin{frame}{4. Gestion et optimisation des coûts}
  \begin{itemize}
  \item Budgétisation efficace
  \item Services cloud, solutions open-source
  \end{itemize}
\end{frame}

\talksection{Exercice}

\begin{frame}{Exercice}
  \textbf{Objectif :} Mettre en pratique les méthodes d'estimation des
  coûts à travers une simulation réaliste d'un projet ML/IA.
\end{frame}

\begin{frame}{Description}
  \begin{itemize}
  \item Choisir un projet
  \item Discuter
    \begin{itemize}
    \item Acquisition des données.
    \item Développement du modèle.
    \item Test et validation.
    \item Déploiement et maintenance.
    \end{itemize}
  \end{itemize}
\end{frame}

\begin{frame}{Tâches}
  \begin{itemize}
  \item Définir les besoins en données et estimer les coûts y associés.
  \item Choisir les technologies et estimer les coûts de développement
  \item Planifier les ressources nécessaires pour le test et la validation.
  \item Estimer les coûts de déploiement et de maintenance sur un an.
  \end{itemize}
\end{frame}

\end{document}
