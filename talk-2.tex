\input talk-header.tex
\title{Machine Learning and AI}
\subtitle{Jour 2 : Développement et Intégration de Projets IA}

\newcommand\hugo[0]{\vfill\prevwork{Hugo Mougard, Machine Learning, CC BY-SA 4.0}}

\begin{document}

\begin{frame}
  \titlepage
\end{frame}


% %%%%%%%%%%%%%%%%%%%%%%%%%%%%%%%%%%%%%%%%%%%%%%%%%%%%%%%%%%%%
\talksection{Doing Data Science}

\begin{frame}[t]
  \frametitle{Perspectives}

  \vspace{1cm}
  \begin{itemize}
  \item Data science is iterative
  \item Start simple, get better
  \end{itemize}

  \only<2>{\blue{Sometimes there's no business case to do more.\\You want to know where that threshold is so you can stop.}}
\end{frame}

\begin{frame}[t]
  \frametitle{Risks}

  \vspace{1cm}
  \begin{itemize}
  \item Nothing is guaranteed\gray{, but competitors are innovating and experimenting}
  \item Examples from past projects can help, but often leads to ``this is different''
  \end{itemize}
\end{frame}

\begin{frame}[t]
  \frametitle{Success}
  \vspace{2cm}
  \centerline{\blue{Think early about how to measure success.}}
  \vspace{1cm}
  \only<2>{The definition of success will evolve.}
\end{frame}

\begin{frame}{Andrew Ng Methodology — Lifecycle of an ML Project}
  Four phases:
  \begin{enumerate}
    \item Project scope definition
    \item Data management
    \item Modeling
    \item Deployment
  \end{enumerate}
  Each phase has its own particular challenges.

  \hugo
\end{frame}

\begin{frame}{1. Project Scope Definition}
  Keypoints:
  \begin{itemize}
    \item Delimitation of the task to be accomplished
    \item Definition of success indicators
    \item Budget in terms of time, personnel, etc.
    \end{itemize}

    \hugo
\end{frame}

\begin{frame}{2. Data Management}
  Keypoints:
  \begin{itemize}
    \item Unbiased data collection method
    \item Clear definition of inputs and outputs
    \item Robust data processing pipeline without training/production disparities
    \item Reproducibility and experiment management
  \end{itemize}

  \hugo
\end{frame}

\begin{frame}{3. Modeling}
  Keypoints:
  \begin{itemize}
    \item Training that takes into account production needs (model size, speed, etc.)
    \item Error analysis, often by significant data slices
    \item Reproducibility and experiment management
  \end{itemize}

  \hugo
\end{frame}

\begin{frame}{4. Deployment}
  Keypoints:
  \begin{itemize}
    \item Scaling
    \item Detection of data and concept drifts
    \item Monitoring
    \item Retraining process
  \end{itemize}

  \hugo
\end{frame}

\begin{frame}{CRISP-DM}
  The cross-industry standard process for data mining.

  The major phases:
  \begin{itemize}
  \item Business Understanding
  \item Data Understanding
  \item Data Preparation
  \item Modeling
  \item Evaluation
  \item Deployment
  \end{itemize}
\end{frame}

\begin{frame}{Technical standpoint — MLOps}
  From a technical standpoint, Data Science projects management is called MLOps:

  \begin{itemize}
    \item Tools bridging the gap from proof of concept to production
    \item Techniques to comply with regulatory obligations
    \item Methods for mitigating ethical issues
  \end{itemize}

  \hugo
\end{frame}

\begin{frame}{State of MLOps}
  It's still a young practice, even if it's grown enormously in ten years.
  \begin{itemize}
    \item Rapid evolution, many competing libraries
    \item Much less stable than DevOps
    \item Adoption much less homogeneous than DevOps
    \item Co-evolution with legislation and regulation
  \end{itemize}
  Given the instability of the domain, broad principles are more
  important than specific techniques.

  \hugo
\end{frame}



% %%%%%%%%%%%%%%%%%%%%%%%%%%%%%%%%%%%%%%%%%%%%%%%%%%%%%%%%%%%%
\talksection{Data}

\begin{frame}{Data Management Phases}
  Two distinct phases:
  \begin{itemize}
    \item Definition and calibration \only<2>{\blue{\small\it (what's needed, how collected, accuracy)}}
    \item Retrieval, labeling and organisation \only<2>{\blue{\small\it (extraction, labeling, organising)}}
  \end{itemize}

  \only<3>{
    \blue{Definition:}
    \begin{itemize}
    \item \blue{Identify Data Requirements}
    \item \blue{Data Sources}
    \item \blue{Data Standards {\small\it(formats, naming conventions, \dots)}}
    \end{itemize}
  }
  \only<4>{
    \blue{Calibration}
    \begin{itemize}
    \item \blue{Data Quality Metrics {\small\it(accuracy, completeness, consistency, timeliness, \dots)}}
    \item \blue{Validation Rules}
    \item \blue{Data Integration}
    \item \blue{Tools and Technologies}
    \end{itemize}
  }
  \only<5>{
    \blue{Retrieval:}
    \begin{itemize}
    \item \blue{Data Extraction}
    \item \blue{Data Aggregation}
    \end{itemize}
  }
  \only<6>{
    \blue{Labeling:}
    \begin{itemize}
    \item \blue{Metadata assignment}
    \item \blue{Data Tagging}
    \item \blue{Annotations}
    \end{itemize}
  }
  \only<7>{
    \blue{Organisation:}
    \begin{itemize}
    \item \blue{Data Structuring}
    \item \blue{Data Storage}
    \item \blue{Data Indexing}
    \item \blue{Data Management}
    \end{itemize}
  }

  \hugo
\end{frame}

\begin{frame}{Objectives}
  We want primarily to address two questions:
  \begin{itemize}
    \item What are the relevant inputs and outputs?
    \item What level of performance can we expect?
  \end{itemize}

  \hugo
\end{frame}


\begin{frame}{\textit{Trash in Trash out} Principle}
  Fundamental Principle of Machine Learning:

  \purple{Data definition is the \textit{central} point of a machine learning system.}

  \hugo
\end{frame}


\begin{frame}{Research vs Industry}
  \begin{description}
    \item[Research] $\text{System} = \text{Data} + \overbrace{\text{Parameters} + \text{Model}}^{\text{Work}}$
    \item[Industry] $\text{System} = \overbrace{\text{Data} + \text{Parameters}}^{\text{Work}} + \text{Model}$
  \end{description}

  \hugo
\end{frame}


\begin{frame}{Definition Example – Translation}
  How to define the output?
  \begin{enumerate}
    \item I was overwhelmed with joy. → J'ai été submergé par la joie.
    \item I was overwhelmed with joy. → Je fus submergé de joie.
    \item I was overwhelmed with joy. → Je fus terrassé par la joie.
  \end{enumerate}

  \hugo
\end{frame}


\begin{frame}{Definition Example – Audio}
  How to define the input?
  \begin{enumerate}
    \item Um… I'll be there in 5 minutes
    \item Um, I'll be there in 5 minutes
    \item I'll be there in 5 minutes
  \end{enumerate}

  \hugo
\end{frame}


\begin{frame}{Definition Example – Identity Fusion}
  How to define the output?
  \begin{itemize}
    \item Martin Durant, 44000, …, <martin@durant.fr>
    \item Martin Durant, 44000, …, <martin.durant@gmail.com>
  \end{itemize}

  \hugo
\end{frame}


\begin{frame}{What is at stake?}
  All these decisions change the function that the model will approximate.

  \hugo
\end{frame}


\begin{frame}{Types of Data}
  Two primary criteria:
  \begin{itemize}
    \item Size of the dataset
    \item Structured or unstructured data
  \end{itemize}

  \hugo
\end{frame}


\begin{frame}{Size of the Dataset}
  How much data we have influences what is important.

  \begin{description}
    \item[Small] Quality of annotations is crucial
    \item[Large] Quality of data processing processes is crucial
  \end{description}

  \bigskip
  \purple{A subset of a large dataset can behave like a small dataset
    (especially a critical slice).}

  \hugo
\end{frame}


\begin{frame}{Structured~/~Unstructured Data}
  Data structure affects what's easy and hard.

  \begin{description}
    \item[Structured] Hard to annotate for humans. Hard to augment.
    \item[Unstructured] Likely easy to annotate for humans. Often augmentable.
  \end{description}

  \hugo
\end{frame}


\begin{frame}{Annotation Guide}
  For annotators:
  \begin{itemize}
    \item Should be as robust as possible
    \item Ideally written by a mix of domain experts / ML
    \item Written iteratively:
      \begin{itemize}
        \item Write a version
        \item Annote
        \item Detect ambiguous points
        \item Write a new version
      \end{itemize}
  \end{itemize}

  \hugo
\end{frame}


\begin{frame}{Coverage of Input Data}
  Data coverage principles:
  \begin{itemize}
    \item All cases to be handled must be represented in the data
    \item All cases to be handled must be represented in sufficient quantities
    \item It's particularly important to avoid discrimination on protected attributes
  \end{itemize}

  \hugo
\end{frame}


\begin{frame}{Calibration}
  Estimating expected performance:
  \begin{itemize}
    \item Bibliographic research on existing approaches
    \item Estimation of human performance
  \end{itemize}

  \hugo
\end{frame}



\begin{frame}{Human Level Performance (HLP)}

  Human Level Performance (HLP) refers to the capability of AI systems
  or models to perform tasks at a level comparable to that of a
  human.

  Achieving HLP means that the AI can handle specific tasks with a
  similar degree of accuracy, efficiency, and reliability as a human
  expert in that domain.

\end{frame}

\begin{frame}{Human Level Performance}

\begin{itemize}
\item \textbf{Benchmark for Performance:} HLP serves as a benchmark to
  estimate the potential maximum performance of a system.
\item \textbf{Bayes' Error Estimation:} It helps in estimating the
  Bayes' error, which is the minimum possible error due to the
  inherent randomness in the data.
\item \textbf{Annotation by External Processes:} HLP is crucial when
  annotations are generated by external processes rather than human
  annotators.
\item \textbf{Unstructured Data:} Particularly relevant for tasks
  involving unstructured data (e.g., images, audio, text).
\item \textbf{Achievable Performance:} Provides an idea of the best
  possible performance that can be achieved by a system.
\end{itemize}

\end{frame}

\begin{frame}{Improving Human Level Performance}

\begin{itemize}
\item \textbf{Underestimating HLP:} Sometimes, HLP is underestimated
  to make it easier for AI models to surpass human performance.
\item \textbf{Impact on Orientation:} Poorly defined HLP can mislead
  the direction of development efforts, resulting in suboptimal
  performance improvements.
\end{itemize}

\end{frame}

\begin{frame}{Improving Human Level Performance — Example}

\textbf{Example of HLP in Action:}
\begin{enumerate}
    \item ``Um\dots I'll be there in 5 minutes.''
    \item ``I'll be there in 5 minutes.''
\end{enumerate}

\begin{itemize}
\item If 80\% of annotators prefer the first transcription and 20\%
  prefer the second, the agreement between two random annotators is
  calculated as: \(0.8^2 + 0.2^2 = 68\%\) agreement.
\item An algorithm that always chooses the most common transcription
  can achieve 80\% agreement.
\item \purple{Significant errors may be obscured by these seemingly
    high agreement rates, highlighting the need for deeper analysis
    beyond superficial gains.}
\end{itemize}

  \hugo
\end{frame}

\begin{frame}{Documentation}
  Things to track during the definition and calibration process:

  \begin{itemize}
    \item Potential biases in the data that you suspect
    \item Real data coverage issues
    \item Regulatory issues related to the data
  \end{itemize}

  \purple{This information is crucial for properly documenting the
    model in the long run.}

  \hugo
\end{frame}



% %%%%%%%%%%%%%%%%%%%%%%%%%%%%%%%%%%%%%%%%%%%%%%%%%%%%%%%%%%%%
\talksection{Data Processing}

\begin{frame}{Data Organization}
  Many options:
  \begin{itemize}
    \item Structuring (schema, description, ...)
    \item Scaling (SQL, NoSQL, distributed file system, ...)
  \end{itemize}

  \hugo
\end{frame}


\begin{frame}{\textit{Feature Store}}
  In an MLOps context, feature stores are often an intermediate step
  between the original source and processing.

  \begin{itemize}
    \item Allows storage optimized for ML
    \item Avoids recomputing the same features
    \item Enables discoverability
  \end{itemize}

  See \textit{Feast} for example.\\
   \url{https://docs.feast.dev/}

  \hugo
\end{frame}


\begin{frame}{Data Acquisition}
  \begin{itemize}
    \item Aim for a short first iteration to get feedback for subsequent phases
    \item List potential sources and their time/money budget
    \item Potentially have the ML team do initial annotation
    \item Define competent profiles
  \end{itemize}

  \hugo
\end{frame}


\begin{frame}{Iterations on Data}
  \begin{itemize}
    \item Order of magnitude: not more than x10 at once
    \item Work jointly on source quality, processes, volume
    \item Very different in prototyping and production phases:\\[5mm]
      \begin{description}
        \item[Prototyping] Gathering enough data to decide go/no-go
        \item[Production] Deep work as the main vector to reach performance ceiling
      \end{description}
  \end{itemize}

  \hugo
\end{frame}


\begin{frame}{Metadata}
  Because data is the heart of an ML system, follow good practices.

  \begin{itemize}
    \item Track provenance
    \item Track transformations
    \item Maintain reproducibility of data acquisition and transformations:\\[2mm]
      \begin{itemize}
        \item Increasingly important for regulation
        \item Essential for debugging
      \end{itemize}
  \end{itemize}

  \hugo
\end{frame}


\begin{frame}{Metadata — Consequences}
  Think about preserving metadata.

  \begin{itemize}
    \item Define data acquisition processes
    \item Define data transformations
    \item Implement dedicated systems
  \end{itemize}

  \purple{This leads to many architecture decisions.}

  \hugo
\end{frame}


\begin{frame}{Processing Pipelines}

  \begin{itemize}
    \item Provide reproducibility and process automation
    \item Often are directed acyclic graphs of operations
    \item Many solutions exist
  \end{itemize}

  \hugo
\end{frame}


\begin{frame}{Desirable Characteristics}
  Strengths of a processing pipeline
  \begin{itemize}
    \item Adaptation to batch (development) as well as real-time (production)
    \item Faithfulness of development/production processing
    \item Scalability
    \item Deployment on various targets
    \item Ease of development
    \item Integration with the ecosystem
  \end{itemize}

  \hugo
\end{frame}


\begin{frame}{Main Processing Pipeline Solutions}
  Strong points of each solution:

  \begin{description}
    \item Deployment options, dev/prod parity, performance\\ \url{https://www.tensorflow.org/tfx}
    \item Kubernetes integration \url{https://www.kubeflow.org/}
    \item Reproducibility and ``low tech'' solution \url{https://dvc.org/}
    \item Flexible, easy to adopt, supported by clouds \url{https://mlflow.org/}
    \item Kubernetes integration \url{https://www.pachyderm.com/}
  \end{description}

  \hugo
\end{frame}



% %%%%%%%%%%%%%%%%%%%%%%%%%%%%%%%%%%%%%%%%%%%%%%%%%%%%%%%%%%%%
\talksection{Modeling}

\begin{frame}{Model Engineering}
  Several key aspects to consider:

  \begin{itemize}
  \item Performance in production
  \item Various deployments with very distinct characteristics
  \item Interpretability
  \item Maintainability
  \item Compliance with regulations (non-discrimination, \dots)
  \end{itemize}

  \hugo
\end{frame}


\begin{frame}{Best Practices}
  In an industrial context, the focus is on the data, not the model~:

  \begin{itemize}
    \item Start simple (heuristic, simple model)
    \item Use the industry standard for the task given your performance and deployment requirements
    \item Improve data first, model second
  \end{itemize}

  \hugo
\end{frame}


\begin{frame}{Model Card}
  Model cards were introduced in \textit{Model Cards for Model Reporting}\\
   \url{https://arxiv.org/abs/1810.03993}

  \begin{itemize}
    \item Specifies ethical and regulatory decisions related to the model
    \item Provides transparency
    \item Goal is informing both the public and internal teams
  \end{itemize}

  \hugo
\end{frame}


\begin{frame}{Model Registry}
  Centralized place to store and retrieve models and associated meta-data~:

  \begin{itemize}
    \item Eases deployment
    \item Solidifies reproducibility
  \end{itemize}

  \url{https://mlflow.org/docs/latest/model-registry.html}

  \hugo
\end{frame}




% %%%%%%%%%%%%%%%%%%%%%%%%%%%%%%%%%%%%%%%%%%%%%%%%%%%%%%%%%%%%
\talksection{Models and Error Analysis}

\begin{frame}{Introduction}
  Error analysis is crucial during development:

  \begin{itemize}
    \item Guides future work
    \item Determines potential progress
  \end{itemize}

  \hugo
\end{frame}


\begin{frame}{Link between error analysis and interpretability}
  \begin{itemize}
    \item Interpretability enables error analysis
    \item Transparency (sometimes a regulatory or functional requirement in production)
    \item Interpretability-performance continuum:
      \begin{itemize}
        \item Interpretable models often insufficient to approximate desired functions
        \item Modern performant models are black boxes
      \end{itemize}
  \end{itemize}

  Often a dilemma in choosing a model.

  \only<2>{\purple{Useless tip: humans are often not explainable.}}

  \hugo
\end{frame}


\begin{frame}{Recommended Approach}
  Some heuristics for error analysis:

  \begin{itemize}
    \item Determine relevant data slices
    \item Estimate model performance and achievable performance on each slice
    \item Prioritize work on slices offering the most impact
  \end{itemize}

  \hugo
\end{frame}


\begin{frame}{Data Slices}
  Idea popularized by Apple in their paper \textit{Overton: A Data System for Monitoring and Improving Machine-Learned Products}
  and \textit{Snorkel} \url{https://www.snorkel.org/}.

  {Excellent Snorkel blog} on the subject.\\
  \url{https://www.snorkel.org/blog/slicing}

  \hugo
\end{frame}


\begin{frame}{Recommended Approach — Example}
  Let's consider working on an image classification system.

  Suppose we distinguish the following slices in our image data:

  \begin{itemize}
    \item Presence of mountains
    \item Presence of humans
    \item Presence of cars
  \end{itemize}

  \hugo
\end{frame}


\begin{frame}{Recommended Approach — Example, Continued}
  We estimate the following performances:

  \begin{tabular}{rcc}
    \hline
    Slice      & HLP  & Model \\
    \hline
    Mountains  & 65\% & 60\%  \\
    Humans     & 96\% & 90\%  \\
    Cars       & 80\% & 40\%  \\
    \hline
  \end{tabular}

  Which slice is most important for improvements in the next iteration?

  \hugo
\end{frame}


\begin{frame}{Recommended Approach — Example, Conclusion}
  Using proportions in the dataset to quantify impact:

  \begin{tabular}{rcccc}
    \hline
    Slice      & HLP  & Model & Proportion & Potential Impact \\
    \hline
    Mountains  & 65\% & 60\%  & 10\%       & 0.5\% \\
    Humans     & 96\% & 90\%  & 85\%       & 5.1\% \\
    Cars       & 80\% & 40\%  & 1\%        & 0.4\% \\
    \hline
  \end{tabular}

  \hugo
\end{frame}


\begin{frame}{Error Analysis}
  Once relevant slices are identified, some options:

  \begin{itemize}
    \item Find explanatory examples of the model
    \item Use a simpler model that globally or locally explains the
      complex model
  \end{itemize}

  \hugo
\end{frame}


\begin{frame}{Explanatory Examples}
  Things to look for:

  \begin{itemize}
    \item \textbf{Prototypes} Representative examples of model behavior
    \item \textbf{Counterfactual examplqes} Modification of existing
      instances to see how prediction evolves
    \item \textbf{Adversarial examples} Counterfactual examples with a
      significant impact on prediction
    \item \textbf{Influential examples} Examples that have had the
      most impact on the model
  \end{itemize}

  \hugo
\end{frame}


\begin{frame}{Explanatory Models}
  Training models:
  \begin{itemize}
    \item Simple
    \item Interpretable (linear regression, simple trees, etc.)
    \item Approximating the complex model
    \item Helping to understand important features
    \item Globally (multiple examples, broad coverage)
    \item Or locally (one example)
  \end{itemize}

  \hugo
\end{frame}


\begin{frame}{Measures}
  Once the analysis is done:

  \begin{itemize}
    \item Data augmentation
    \item Feature engineering
    \item Exploration of new parameters
  \end{itemize}

  \hugo
\end{frame}


\begin{frame}{Further Reading}
  \textit{Interpretable Machine Learning} book\\
  \url{https://christophm.github.io/interpretable-ml-book/}

  \hugo
\end{frame}




% %%%%%%%%%%%%%%%%%%%%%%%%%%%%%%%%%%%%%%%%%%%%%%%%%%%%%%%%%%%%
\talksection{Intro to Production Deployment}

\begin{frame}{Deployment Challenges}
  \begin{itemize}
    \item Providing a trained model to a diverse and large crowd
    \item With ML performance observed in quality tests
    \item With good classical performance (latency, throughput, \dots)
  \end{itemize}

  \hugo
\end{frame}


\begin{frame}{Common Pitfalls}
  What can go wrong?  \textit{(Or: what often goes wrong?)}

  \begin{itemize}
    \item Different code in development vs production
    \item Poor dependency management
    \item Inadequate or inappropriate architecture
  \end{itemize}

  \hugo
\end{frame}


% %%%%%%%%%%%%%%%%%%%%%%%%%%%%%%%%%%%%%%%%%%%%%%%%%%%%%%%%%%%%
\talksection{Deployment Best Practices}

\begin{frame}{Deployment Strategies}
  Several criteria allow for choosing a deployment strategy:

  \begin{itemize}
    \item Service traffic
    \item Audience (public, internal, other services, \dots)
    \item Deployment frequency
    \item \dots
  \end{itemize}

  Key concepts:
  \begin{itemize}
    \item Progressive deployment
    \item Rollback (ability to return to the previous state of the system)
  \end{itemize}

  \hugo
\end{frame}


\begin{frame}{Shadow Deployment}
  \begin{itemize}
    \item Deployment of the model alongside the existing system
    \item Model outputs are not used by the application
    \item Analysis of model outputs and decision to continue deployment or not
  \end{itemize}

  \hugo
\end{frame}


\begin{frame}{Blue/Green Deployment}
  \begin{itemize}
    \item Deployment of the new model (green) alongside the existing system (blue)
    \item When tests are successful on the green system, traffic is redirected there
    \item Allows for rollback if issues arise
  \end{itemize}

  \hugo
\end{frame}


\begin{frame}{Canary Deployment}
  \begin{itemize}
    \item Similar to blue/green, two parallel systems
    \item Gradual ramp-up of the green system
  \end{itemize}

  \hugo
\end{frame}


\begin{frame}{Implementation}
  The two most popular options:
  \begin{itemize}
    \item Kubernetes + istio
    \item Internal tooling
  \end{itemize}

  \hugo
\end{frame}


\begin{frame}{CI/CD}
  The launch of these deployments can be done via git tags and operations in CI/CD.

  This is the gitops approach with gto project from DVC, for example\\[5mm]
  \url{https://www.atlassian.com/git/tutorials/gitops}
  \url{https://dvc.org/doc/gto}

  \hugo
\end{frame}





% %%%%%%%%%%%%%%%%%%%%%%%%%%%%%%%%%%%%%%%%%%%%%%%%%%%%%%%%%%%%
\talksection{Monitoring}

\begin{frame}{Challenges}
  \begin{itemize}
    \item Model drift
    \item Ensuring proper behavior of a model
    \item Knowing when to retrain a model
    \item Understanding the data of a model
  \end{itemize}

  \hugo
\end{frame}


\begin{frame}{Common Issues}

  \textbf{Data Drift}: the evolution of the input data distribution.

  \purple{Gradually makes models obsolete, so need to retrain.}

  \hugo
\end{frame}


\begin{frame}{Common Issues}
  \textbf{Concept Drift}: the evolution of the correspondence between
  outputs and inputs.

  \purple{Gradually makes models obsolete, so need to retrain.}

  \hugo
\end{frame}


\begin{frame}{Speed of Data Evolution}
  Different data sets evolve at different rates.

  \begin{itemize}
    \item \textbf{User Data} often changes slowly but fundamentally
    \item \textbf{Enterprise Data} often changes more markedly and abruptly
  \end{itemize}

  \hugo
\end{frame}


\begin{frame}{Common Issues}
  \textbf{Data Quality:}

  \begin{itemize}
    \item Missing values
    \item Outliers
    \item Schema changes
    \item \dots
  \end{itemize}

  \hugo
\end{frame}


\begin{frame}{Tools — Test Suites}
  Automated execution for detection of all these challenges:
  \begin{itemize}
    \item Data drift
    \item Concept drift
    \item Poor data quality
    \item Drop in prediction performance
  \end{itemize}
  The most widely used library for this is \texttt{evidently}.

  \hugo
\end{frame}


\begin{frame}{Tools — Dashboard}
  The human-facing counterpart of test suites:

  \begin{itemize}
    \item To fuel discussion on data within the Machine Learning team
    \item To complement more general dashboards
  \end{itemize}

  \hugo
\end{frame}


\begin{frame}{Classic Monitoring}
  In addition to these Machine Learning-specific issues, a production system needs usual monitoring.

  \textit{E.g.}, the four "golden signals" in Site Reliability Engineering:

  \begin{itemize}
    \item Latency
    \item Traffic
    \item Errors
    \item Saturation
  \end{itemize}

  \hugo
\end{frame}


\begin{frame}{Implementation}
  Several solutions are common:

  \begin{itemize}
    \item {Evidently} \url{https://www.evidentlyai.com/}
    \item {Seldon Core} \url{https://www.seldon.io/solutions/core-plus}
    \item {TFX ExampleValidator} \url{https://www.tensorflow.org/tfx/guide/exampleval}
    \item {Great Expectations} \url{https://docs.greatexpectations.io/docs/home/}
  \end{itemize}

  \hugo
\end{frame}




% %%%%%%%%%%%%%%%%%%%%%%%%%%%%%%%%%%%%%%%%%%%%%%%%%%%%%%%%%%%%
\talksection{Case Study}

\begin{frame}
  \vphrase{Let's talk about you}
\end{frame}


\end{document}
