\input ../talk-header.tex
\date{octobre 2024}
\title{Machine Learning et IA}
\subtitle{Modèles, algorithmes et intuition}

\begin{document}

\begin{frame}
  \titlepage
\end{frame}

\talksection{Bienvenue}

\begin{frame}{Structure}
  \only<1>{
    Une matinée
    \begin{itemize}
    \item  17 octobre 2024
    \end{itemize}
  }
  \only<2>{
    \begin{itemize}
    \item Email \url{jeff@p27.eu}
    \item Github : \url{https://github.com/JeffAbrahamson/ML-diva-beapp}
    \end{itemize}
  }
  \only<3>{
    Un peu en boucle :
    \begin{itemize}
    \item Modèles
    \item Algorithmes
    \item Exemples
    \end{itemize}
  }
\end{frame}

\talksection{Go}

\begin{frame}{Le modèle le plus simple}
  Quel est le problème le plus simple imaginable ?

  Quel est le modèle le plus simple imaginable ?
  \only<2->{

    \blue{Un constant}
  }
  \only<3->{

    \purple{Par exemple ?}
  }
  \only<4->{

    \small{\it La hauteur d'une porte.  Deux cas{\only<4>{\dots}\only<5->{ : \purple{\it design, découverte}}}}
  }
  \only<6->{

    Par exemple,
    \begin{itemize}
    \item Moyen
    \item Médian
    \end{itemize}
  }
\end{frame}

\begin{frame}{Un modèle toujours simple}
  Mais un peu plus complexe.

  Le modèle \only<1>{?}\only<2->{: \blue{linéaire}}
  \only<2->{

    \purple{Des exemples ?}
  }
\end{frame}

\begin{frame}{Et si ce n'est pas linéaire ?}
  \only<1>{Comment s'adapter ?}
  \only<2->{1. Commencer avec linéaire, est-ce que ça ajoute la valeur nécessaire ?}
  \only<3->{

    2. Sinon, ajouter un nouveau modèle aux résidus.

    \purple{\it Regarder les résidus, c'est quand même toujours une bonne idée.}
  }
  \only<4->{

    Exemple A : (presse) Changement d'UX, plus de lecture ?
    \purple{\it(penser métriques, critères)}
  }
  \only<5->{

    Exemple B : (IOT) Changement de X, plus d'autonomie ?
    \purple{\it(penser métriques, critères)}
  }
\end{frame}

\begin{frame}{Vraiment pas linéaire}
  Distinguons entre pas linéaire parce que bombé \textit{(linéaire
    peut marcher toute de même)}

  \red{--- et ---}

  rugueux, petites vagues complexes, beaucoup de minima et maxima
  locaux \textit{(pas de chance avec linéaire)}

  \only<2->{\purple{Rugueux peut bénéficier d'un modèle capable de
      trouver un séparateur plus complèxe.}
}
\only<3->{

  Options :
  \begin{itemize}
  \item SVM avec RBF (assez simple à mettre en oeuvre)
  \item Random forest (meilleure explicabilité, pour un certain esprit)
  \item ANN or gradient boosted trees (plus de boulot garanti, mais souvent meilleur)
  \end{itemize}
  }
\end{frame}

\begin{frame}{Exercice}
  \underbar{Presse}

  J'utilise un modèle génératif de Hugging Face afin de créer des
  résumé de 1--2~lignes de chaque article.  J'affiche le résumé en
  haut de l'article, façon chapô.

  \purple{Questions :}
  \only<2->{
  \begin{itemize}
  \item Est-ce que les lecteurs lisent plus d'articles ?
    \only<3->{\item Est-ce que les lecteurs cliquent plus de publicités ?}
    \only<4->{\item Est-ce qu'un lecteur qui clique deux articles aura plus de chance d'en cliquer~5 ?}
    \only<5->{\item Est-ce que le longueur de l'abstrait change la réponse ?}
  \end{itemize}
  }
\end{frame}

\begin{frame}{Exercice}
  \underbar{E-commerce}

  Je teste un nouveau système de recommandation.

  \purple{Questions :}
  \only<2->{
    \begin{itemize}
    \item Est-ce que le changement augmente le CTR ?
    \only<3->{\item Est-ce que le changement augmente revenu ?}
    \only<4->{\item Est-ce que le changement augmente le taux de retour ?}
    \end{itemize}
  }
\end{frame}

\begin{frame}{Exercice}
  \underbar{Détection de fraude}

  \only<2->{Définir le problème, quel surface ou frontière de décision
    pensez-vous avoir créer ?}

  \only<3->{Quels algorithmes vous semblent pertinents (et pourquoi) ?}

  \only<4->{Quels problèmes sont pertinents à considérer une fois que
    vous commencez à travailler ?}
\end{frame}

\begin{frame}{Exercice}
  \underbar{Vos problèmes}
\end{frame}

\end{document}
